En el Laboratorio de Fluidos de la Universidad, se utiliza el túnel de viento para realizar el contraste de anemómetros y experimentos para distintas materias. Gran parte de estas aplicaciones requieren que se conozca la velocidad del fluido (aire). Por lo tanto, la presión diferencial del tubo pitot, humedad, presión atmosférica y temperatura son variables requeridas para lograr estimarla con mayor precisión.
Antes del desarrollo de éste proyecto, cada una de éstas variables se medían de forma manual, con sus respectivos instrumentos, para luego ingresar estos valores a una tabla de cálculo para obtener una estimación de la velocidad del aire.\\

En sus comienzos, la realización de distintas experiencias en el túnel de viento se realizaban utilizando un control de velocidad de lazo abierto, que consistía en la modificación de forma discreta de la resistencia rotórica del motor, que es una máquina asíncrona de inducción de rotor bobinado, con lo que se lograba cambiar la velocidad del aire también en forma discreta. Desde principios de 2020 y hasta la actualidad se utiliza un variador de velocidad de la marca \textbf{Long Shenq}, con él se obtiene un control continuo de la velocidad, aunque todavía el control es de lazo abierto. \\

Realizar este proceso de forma manual, se torna engorroso y poco práctico para la realización de varias mediciones por lo que se realiza este trabajo final de carrera para realizar la \textit{Automatización del Túnel de Viento de la UNPSJB}.

\newpage