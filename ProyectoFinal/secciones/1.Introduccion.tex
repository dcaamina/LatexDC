Actualmente, en el Laboratorio de Fluidos de la Universidad, se utiliza el Túnel del Viento para realizar el contraste de anemómetros y experimentos para distintas materias. Gran parte de estas aplicaciones requieren que se conozca la velocidad del fluido. Por lo tanto, variación de presión, humedad, presión atmosférica y temperatura son variables requeridas para lograr estimarla con mayor precisión.
Cada variable debe ser medida de forma manual con sus respectivos instrumentos para luego ingresar estos valores a una tabla (generada de forma estadística) y obtener una estimación de la velocidad del fluido.
Para realizar distintas mediciones, se utilizaba un control de velocidad a lazo abierto en el que se modifica la resistencia del motor, cambiando la velocidad del aire en pasos discretos. Actualmente, desde principios del año 2020 se utiliza un variador de velocidad de la marca \textbf{Long Shenq}.
Realizar este proceso de forma manual, se torna engorroso y poco práctico para la realización de varias mediciones.

\newpage