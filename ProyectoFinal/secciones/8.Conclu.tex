Siempre que se habla de control de procesos industriales, los PLC son los dispositivos más adecuados para el desarrollo de sistemas de este tipo. Para nosotros fue un desafío utilizar un microcontrolador ($\mu$C) para llevar a cabo el control de velocidad del túnel de nuestra Universidad.

Al utilizar un $\mu$C no se tiene la robustez de un dispositivo industrial, pero sin embargo, logramos una comunicación estable y un control eficaz sobre el variador de velocidad. Además, al disponer de esta opción, los costos de diseño e implementación son más económicos contra el empleo de dispositivos industriales.

Si bien es cierto que teníamos conocimientos básicos en el uso de herramientas de programación, el proyecto de automatización del túnel fue un aprendizaje que sumó a nuestros conocimientos nuevos lenguajes de programación, técnicas de trabajo y adicionalmente, saberes sobre el funcionamiento de los dispositivos que realizan las acciones sobre el proceso: motores, variadores de velocidad, señales analógicas y digitales, etc. 

La realización de una aplicación que permite la adquisición de datos,  la visualización de los mismos en tiempo real, el guardado de éstos en tablas, y la realización de gráficos de los datos guardados, hacen al sistema flexible tanto para la realización de los ensayos en el túnel de viento, como para el análisis posterior de los datos medidos. Además, se posibilitó el manejo remoto del variador de velocidad, y un control de lazo cerrado, en reemplazo de la utilización del panel frontal del variador, y del control de lazo abierto.

Como resultado final, se obtuvo un control de velocidad, en donde las mediciones obtenidas fueron cercanas a los valores estimados por el Laboratorio Mecánica de Fluidos, lo que ayudaría a minimizar la cantidad de procesos para la determinación de velocidad durante la contrastación de anemómetros. Además, nuestro controlador permitiría realizar nuevas experiencias de laboratorio de asignaturas afines a la temática, ya que se podrían generar cambios de velocidad controlados, perfiles de viento determinados, simular ráfagas, entre otras aplicaciones.

