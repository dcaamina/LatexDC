Siempre que se habla de control de procesos industriales, los PLC son los dispositivos más adecuados para el desarrollo de sistemas de este tipo. Para nosotros fue un desafío utilizar un microcontrolador para llevar a cabo el control de velocidad del túnel de nuestra Universidad.

Al utilizar un $\mu$C no se tiene la robustez que un dispositivo industrial posee, pero sin embargo, logramos una comunicación estable y un control eficaz sobre el variador de velocidad. Además, al disponer de esta opción, los costos de diseño e implementación son más económicos contra el empleo de dispositivos industriales.

Por más que teníamos noción sobre las herramientas de programación, el proyecto de automatización del túnel fue un nuevo aprendizaje que sumó a nuestros conocimientos nuevos lenguajes de programación, técnicas de trabajo y adicionalmente, saberes sobre el funcionamiento de los dispositivos que realizan las acciones sobre el proceso: motores, variadores de velocidad, señales analógicas y digitales, etc. 

Al realizar una aplicación para visualizar datos relevantes y un gráfico en tiempo real, esta interfaz nos permite realizar la adquisición de datos para guardarlos en tablas y ser trabajados en un posterior análisis. La misma permitió tener un manejo remoto del variador de velocidad.

Como resultado final, se obtuvo un control de velocidad, en donde las mediciones obtenidas fueron cercanas a los valores estimados por el Laboratorio Mecánica de Fluidos, lo que ayudaría a minimizar la cantidad de procesos para la determinación de velocidad durante la contrastación de anemómetros. Además, nuestro controlador permitiría realizar nuevas experiencias de laboratorios por materias afines a la materia, ya que se podrían generar cambios de velocidad controlados, tipo ráfagas.

