\begin{tcolorbox}[colback=blue!5!white,colframe=blue!75!black,title=I2C]
	Es un puerto y protocolo de comunicación serial, define la trama de datos y las conexiones físicas para transferir bits entre 2 dispositivos digitales. El puerto incluye dos cables de comunicación, SDA (Datos seriales) y SCL (reloj serial). Además el protocolo permite conectar hasta 127 dispositivos esclavos con esas dos líneas, con hasta velocidades de 100, 400 y 1000 kbits/s. \end{tcolorbox}

\begin{tcolorbox}[colback=blue!5!white,colframe=blue!75!black,title=Arduino] El\textbf{ Arduino Uno} es una placa de microcontrolador de código abierto basado en el microchip \textbf{Atmega328P} y desarrollado por \textbf{Arduino}. La placa está equipada con conjuntos de pines de E/S digitales y analógicas que pueden conectarse a varias placas de expansión y otros circuitos.
   \end{tcolorbox}

	

Para la elección del microcontrolador se analizó varios dispositivos, como por ejemplo \textbf{Arduino MEGA}, \textbf{Arduino NANO}, \textbf{EDUCIAA}, entre otros. Se llegó a la conclusión que Arduino UNO era un microcontrolador de bajo costo, baja complejidad en la programación, con librerías de diversos sensores y que cumplía con las prestaciones necesarias (señal de PWM, comunicación \textbf{I2C}, entradas analógicas y digitales).


Para comenzar con las pruebas, lo primero que se realizó es la conexión de diversos sensores en una protoboard. Se obtuvieron datos para elegir cuál  convenía utilizar, ya que se hicieron pruebas con varios sensores de temperatura, humedad y presión atmosférica.


Cabe destacar que para la comunicación de estos sensores se utilizó \textbf{comunicación I2C}, que permitió, a través de dos líneas de comunicación, conectar, en una primera instancia,  5 elementos al mismo bus.

Las variables que se necesitó medir correspondían a diferencia de presión, temperatura, humedad y presión atmosférica, para esto se consiguió diversos sensores para realizar pruebas eligiendo elementos que generaban la menor variación de THP respecto a los instrumentos que posee el laboratorio y estos fueron los siguientes:

\begin{itemize}
	\item  \textbf{BME280:} Sensor de presión atmosférica, temperatura y humedad relativa.
	\item \textbf{SI7021:} Sensor de temperatura y humedad relativa.
	\item \textbf{MPXV7002:} Sensor de presión diferencial.
	\item \textbf{ADS1115:} Convertidor analógico digital 16bits utilizado en conjunto con MPXV7002.
\end{itemize}


Se realizaron diversas pruebas que sirvieron para probar y corroborar el funcionamiento del \textbf{MPXV7002}, un sensor de diferencia de presión de alto costo en el país. Este sensor es capaz de medir de -2kPa a 2kPa en un rango de 4V (0.5 a 4.5V), para este proyecto se utilizó la parte positiva de los valores por lo que la resolución estaba establecida por los 2V de rango y 2kPa. Estos datos se ingresaron al programa \textbf{Arduino} a través de un conversor \textbf{ADS 1115} con comunicación I2C.

No se utilizó el ADC interno del microcontrolador Arduino Uno ya que es de 10 bits y el externo de 15 bits más un bit de signo, este mismo posee un amplificador de ganancia programable (PGA) que establece la escala completa, es decir, indica el valor de referencia. En Arduino este valor viene determinado por el voltaje de referencia que en el caso de Arduino UNO es 5V. En el ADS1115 lo establece el PGA. Por defecto este valor de referencia es ±6,144 V, quiere decir que el valor de 32.677 (valor máximo con 15-bit) corresponde a 6,144 V.

Ejemplo Factor de escala de Arduino UNO:
\begin{center}
	\begin{math}Factor\;de\;escala=\frac{5\;V}{1023}=\;0,0048875\;V\;=\;4,88\;mV
	\end{math}
\end{center}
Ejemplo Factor de escala de ADS115:
\begin{center}
	\begin{math}Factor\;de\;escala=\;\frac{6.144\;V}{32677}=\;0,0001875\;V\;=\;0,1875\;mV
	\end{math}
\end{center}

En la siguiente tabla se tiene un resumen de los valores posibles de factor de escala para ADS1115.
\begin{table}[h]
	\centering
	
	\begin{tabular}{|c|c|c|}
		\hline
		\textbf{PGA} & \textbf{Referencia (V)} & \textbf{Factor de Escala (mV)} \\ \hline
		2/3          & 6,144                   & 0,1875                         \\ \hline
		1            & 4,096                   & 0,125                          \\ \hline
		4            & 1,024                   & 0,0312                         \\ \hline
		8            & 0,512                   & 0,0156                         \\ \hline
		16           & 0,256                   & 0,0078                         \\ \hline
	\end{tabular}
\caption{Factor de escala}
\end{table}

Aunque la mayor referencia sea de 6,1444 V, el ADS1115 sólo puede medir el valor de alimentación (VDD) más 0,3 V por sus pines analógicos. \hl{ver porq} pin ingresa la señal en proteus. Por lo tanto, si alimentamos el ADS1115 con 5V, sólo podremos medir tensiones por los pines analógicos hasta 5V + 0,3V es decir, 5,3V. Lo mismo ocurriría si alimentamos con 3,3V. En este caso el máximo sería 3,6V. \hl{(con que} alimentamos? ver en proteus)\\
En el caso de este proyecto, se utilizó un valor de PGA de 1, tomando como voltaje de referencia 4,096V.
\begin{center}
	\begin{math}
		Factor\;de\;escala=\;\frac{4,096\;V}{32677}=\;0,000125\;V\;=\;0,125\;mV
	\end{math}
\end{center}

\subsubsection{Direcciones I2C}

\subsection{Ecuación velocidad del aire}
El Laboratorio de Mecánica de Fluidos, antes de comenzar con este proyecto utilizaba un archivo Excel para hacer la corrección de la velocidad del aire. En este se calcula matemáticamente la densidad del aire en función de la presión, temperatura y humedad atmosférica.
%	densidad_funcion_P_T_H.pdf  dentro del drive
\begin{align}
	\rho&=\frac{3,48353\;10^{-3}\;kg\;K\;J^{-1}\;\cdot p\cdot\;(1-0,378\;\cdot\;x_v)}{Z\;\cdot\;T} \label{ec_den}\\
	x_v&=\frac{(\alpha+\beta\cdot p+\gamma\cdot t^2)\cdot(1Pa\cdot\;e^{AT^2+BT+C+D/T)})\cdot h/100}p\\
	Z&=\;1-\frac pT\cdot\lbrack a_0+a_1t+a_0t^2+(b_0+b_1t)x_v+(c_0+c_1t)x_v^2\rbrack+(d+x_v^2)\frac{p^2}{T^2}
\end{align}

dónde:\\
- \textbf{p } [Pa] presión atmosférica medida,\\
- \textbf{t } [$^{\circ}$C] temperatura medida,\\
- \textbf{T } [K] temperatura absoluta (\textbf{T}=\textbf{t} + 273,15 \textbf{K})\\
- \textbf{h } [\%] humedad relativa medida,\\
- y constantes \textbf{A,B,C,D,} $\boldsymbol{\alpha , \beta  , \gamma , a_0, a_1 ,a_2 ,b_0  ,b_1 , c_0 , c_1, D. }$


Finalmente, la ecuación anteriormente nombrada es utilizada para el cálculo final de la velocidad del aire dentro del mismo archivo.
\begin{equation}
	\triangle P=\frac{v^2\rho}2\rightarrow\;\;v=\sqrt{\frac{2\;.\;\triangle P\;}\rho}
	\label{ec_aire}
\end{equation}

Tanto las ecuaciones de densidad ec.(\ref{ec_den})
 y la ecuación del cálculo de velocidad  ec.(\ref{ec_aire})
  se desarrolló dentro del programa de \textbf{Arduino} para observar como dato final la velocidad del aire.


\subsubsection{Comparación de densidades} \label{cap:densidades}

Una vez que se tuvo seguridad con los sensores elegidos se procedió a realizar una placa con estos elementos para que no se desconecten y no se produzcan errores como solía suceder mientras estaban en la protoboard (Figura \ref{fig:sensoresa}). \\
\begin{figure}[htb]
	\centering
	\includegraphics[scale=0.8]{placa_sensores.pdf}
	\captionof{figure}{Esquema de placa con sensores utilizados}
	\label{fig:sensoresa}
\end{figure}

A raíz de varias mediciones, durante distintos días, y realizando el contraste con el instrumento \textbf{TESTO 435} se decidió sacar de la caja donde estaban a los sensores de temperatura y humedad para que luego el cálculo de la velocidad del aire no esté desfasado, ya que esta caja utilizada generaba un ambiente distinto al real dentro del laboratorio. Al colocar el sensor del lado externo, se realizaron tomas de valores en distintos momentos. Con ambos conjuntos de datos se procedió a obtener los valores de la densidad calculados con la fórmula. Estos datos están expresados en la tabla \ref{densicalc}. 


\begin{table}[b]
	\centering
	\begin{tabular}{ll|l|l|l|l|l|l|l|l|l|}
		\cline{3-11}
		\multicolumn{2}{c}{} & \multicolumn{2}{|c|}{\textbf{T [$^{\circ}$$C$]}} & \multicolumn{2}{c|}{\textbf{H [$\%$] }} & \multicolumn{2}{c|}{\textbf{P [$Pa$]}} &  \multicolumn{2}{c|}{\textbf{\begin{tabular}[c]{@{}c@{}}Dens. calculada\\ {[}$kg/m^3${]}\end{tabular}}} & \multicolumn{1}{c|}{\multirow{2}{*}{\textbf{$e_r$ [$\%$]}}} \\ \cline{1-10}
		\multicolumn{1}{|c|}{\textbf{Fecha}} & \multicolumn{1}{c|}{\textbf{Obs.}} & \multicolumn{1}{c|}{\textbf{I}} & \multicolumn{1}{c|}{\textbf{S}} & \multicolumn{1}{c|}{\textbf{I}} & \multicolumn{1}{c|}{\textbf{S}} & \multicolumn{1}{c|}{\textbf{I}} & \multicolumn{1}{c|}{\textbf{S}} & \multicolumn{1}{c|}{\textbf{I}} & \multicolumn{1}{c|}{\textbf{S}} & \multicolumn{1}{c|}{} \\ \hline
		\multicolumn{1}{|l|}{29-abr} & interior & 19,4 & 19,3 & 43,5 & 38,5 & 100400 & 100450 & 1,1916 & 1,1931 & 0,128 \\ \hline
		\multicolumn{1}{|l|}{14-may} & interior & 16,1 & 18,4 & 54,6 & 43,5 & 101590 & 101570 & 1,2195 & 1,2099 & 0,781 \\ \hline
		\multicolumn{1}{|l|}{14-may} & interior & 16,5 & 18,9 & 53,7 & 42,5 & 101559 & 101559 & 1,2173 & 1,2077 & 0,794 \\ \hline
		\multicolumn{1}{|l|}{17-jun} & exterior & 15,2 & 15,5 & 54,3 & 48,6 & 103100 & 103090 & 1,2418 & 1,2404 & 0,109 \\ \hline
		\multicolumn{1}{|l|}{17-jun} & exterior & 13,8 & 15,1 & 59,2 & 52,4 & 102970 & 102910 & 1,2463 & 1,2397 & 0,527 \\ \hline
		\multicolumn{1}{|l|}{07-jul} & interior & 17,6 & 18,8 & 43,3 & 34,8 & 100260 & 100261 & 1,1978 & 1,1933 & 0,375 \\ \hline
	\end{tabular}
\caption{Comparación de densidades calculadas}
\label{densicalc}
\end{table}

	


Referencias de la tabla:
\begin{itemize}
	\item \textbf{I}: Instrumentos- Datos obtenidos con los instrumentos del Laboratorio de fluidos.
	\item \textbf{S}: Sensores- Datos obtenidos a partir de la medición con los sensores utilizados en conjunto con Arduino.
	\item \textbf{interior}- Mientras se realizaron las mediciones el sensor SH21 se encontraba dentro de la caja dónde estaba el Arduino y la placa reguladora.
	\item \textbf{exterior}- Las mediciones se realizaron con el sensor SH21 en el lado exterior sin que fuera afectado por el calentamiento del Arduino y la placa reguladora.
\end{itemize}


Tomando como valor verdadero los valores de THP medidos con los instrumentos calibrados, se puede observar que el error relativo es menor al 1\%.


\subsection{Características del flujo}
Una inspección cuidadosa del flujo en una tubería revela que el flujo de fluidos es currentilíneo a bajas velocidades, pero se vuelve
caótico conforme la velocidad aumenta por arriba de un valor crítico. Se dice que el régimen de flujo en el primer caso es laminar, y se caracteriza por líneas de corriente suaves y movimiento sumamente
ordenado; mientras que en el segundo caso es turbulento, y se caracteriza por
fluctuaciones de velocidad y movimiento también desordenado. La transición
de flujo laminar a turbulento no ocurre repentinamente; más bien, sucede sobre
cierta región en la que el flujo fluctúa entre flujos laminar y turbulento antes de
volverse totalmente turbulento. La mayoría de los flujos que se encuentran en la
práctica son turbulentos. 

\subsubsection{Cálculo de número de Reynolds}
\begin{tcolorbox}[colback=blue!5!white,colframe=blue!75!black,title=Número de Reynolds]
	El número de Reynolds (Re) es un número adimensional utilizado en mecánica de fluidos para caracterizar el movimiento de un fluido. Su valor indica si el flujo sigue un modelo laminar o turbulento.
\end{tcolorbox}
	El número de Reynolds fue calculado a partir de la ecuación \ref{reyn}, que relaciona las fuerzas inerciales y las fuerzas viscosas. 
\begin{equation}	
	R_e=\frac{\rho\;D_h\;v}\mu
	\label{reyn}
\end{equation}	
dónde:\\
- \textbf{$\rho$ } [$kg/m^{3}$] densidad del aire \\
- \textbf{$v$ } [$m/s$] velocidad del aire\\
- \textbf{$D_h$ } [$m$] diametro hidráulico\\
- \textbf{$\mu$ } [$kg/m.s$] viscosidad dinámica del aire\\


La viscosidad dinámica del fluido es un valor obtenido del anexo del libro \textit{Mécanica de Fluidos} \cite{yunus2006mecanica}, mientras que el valor de densidad fue calculado con los valores de temperatura, humedad y presión de un día dado como se muestra en la sección \ref{cap:densidades}.
El díametro hidráuliaco de una tubería ciracular coincide con el díametro interno, en tanto que para una tubería de sección rectangular, este es dependiente del valor de sus lados (Figura \ref{fig:rey2}) \cite{licci2020estudio}.

El número de Reynolds se calculó con una velocidad de 6m/s, al ser proporcional a la velocidad si esta aumenta el valor será mayor.

\begin{equation}	
	R_e=\frac{1,193\;kg/m^{3}\;0,86\;m\;6\;m/s}{0,000018\;kg/m.s}=345521
	\label{reyn2}
\end{equation}	

\begin{figure}[htb]
	\centering
	\includegraphics[scale=0.45]{rey2.jpg}
	\captionof{figure}{Diámetro hidráulico}
	\label{fig:rey2}
\end{figure}


\begin{figure}[htb]
	\centering
	\includegraphics[scale=0.9]{camaraensayos.jpg}
	\captionof{figure}{Dimensiones de la cámara de ensayos en mm}
	\label{fig:camens}
\end{figure}



Por definición, un flujo es turbulento si el valor del número de Reynolds es mayor a 4000. Por lo que se observa claramente, que para los valores de velocidad utilizados en el túnel de viento este valor siempre es mayor correspondiendo a un flujo turbulento.

\subsubsection{Flujo turbulento}
\begin{tcolorbox}[colback=blue!5!white,colframe=blue!75!black,title=Flujo turbulento]
	Este se caracteriza por fluctuaciones aleatorias y rápidas de regiones giratorias de fluido, llamadas remolinos a lo largo del fujo. En flujo laminar, las partículas fluyen en orden a lo largo de trayectorias, en cambio, en flujo turbulento, los molinos giratorios transportan masa, cantidad de movimiento y energía a otras regiones del flujo. \cite{yunus2006mecanica}
	
\end{tcolorbox}

Aun cuando el flujo promedio sea estacionario, el movimiento de remolinos en flujo turbulento provoca fluctuaciones importantes en los valores de velocidad, temperatura, presión e incluso densidad (en flujo compresible). Las posibles causas de la formación de estos remolinos se pueden deber a la construcción del túnel de viento. Este, posee una rejilla de entrada en forma de panal de abeja que, al estar cerca de la cámara de ensayos, no logra realizar una correcta formación de flujo uniforme y homogéneo, otra causa posible es la existencia de orificios en la parte inferior de la cámara que son utilizados para la colocación de instrumentos, pero esto genera la no uniformidad del flujo (Figura \ref{fig:remolinos}).
\begin{figure}[htb]
	\centering
	\includegraphics[scale=0.6]{remolinos.jpg}
	\captionof{figure}{Perforaciones dentro de la cámara de ensayos
	 y "Panal de abeja"}
	\label{fig:remolinos}
\end{figure}




La figura \ref{fig:fluct} muestra, en este caso, las fluctuaciones alrededor de una velocidad promedio dentro de un tiempo específico. El valor promedio de una propiedad en alguna posición se determina cuando se promedia sobre un intervalo de tiempo que sea suficientemente grande, de modo que el valor promediado en tiempo se estabilice en una constante. En consecuencia, la fluctuación promediada en tiempo es cero.

\begin{figure}[htb]
	\centering
	\includegraphics[scale=0.5]{fluct.png}
	\captionof{figure}{Fluctuaciones}
	\label{fig:fluct}
\end{figure}

Para corroborar esta propiedad, se usaron datos medidos y a través de un código generado en Matlab y se corroboró el promedio de las fluctuaciones que tiende a cero.
\begin{figure}[h!tb]
	\centering
	\includegraphics[scale=0.5]{fluctua.png}
	\captionof{figure}{Fluctuaciones en velocidades medidas}
	\label{fig:fluct2}
\end{figure}

%C:\Users\glori\Desktop\DANIELA\VISUAL_DANI\Proyecto_Final_Tunel\Pruebas22.03.21


\subsection{Filtros}
\begin{tcolorbox}[colback=blue!5!white,colframe=blue!75!black,title=Mediana]
	Es una técnica de filtrado digital no lineal que suele utilizarse para eliminar el ruido de una imagen o señal. La mediana es el número que está justo en el medio de un conjunto de datos ordenados de menor a mayor o de mayor a menor.
	La idea principal del filtro de mediana es recorrer la señal entrada, sustituyendo cada dato por la mediana de una ventana de “N” datos.
\end{tcolorbox}

Una vez que se procedió a tomar diversos valores, se notó necesario la implementación de un filtro. Para esto se utilizaron varias librerías de \textbf{Arduino} para generar distintas pruebas (Figura \ref{fig:filtros}).

\begin{figure}[!h]
	\centering
	\includegraphics[scale=0.5]{filtros.png}
	\captionof{figure}{Datos y diversos filtros}
	\label{fig:filtros}
\end{figure}

Luego de varias pruebas, se eligió un filtro de mediana (Figura \ref{fig:filtrosm}) con una ventana 40, que producía menor ruido en la velocidad del aire. Se implementó en el programa de \textbf{Arduino} utilizando la una función preestablecida.

\begin{figure}[htb]
	\centering
	\includegraphics[scale=0.5]{filtro mediana.png}
	\captionof{figure}{Datos y filtro mediana}
	\label{fig:filtrosm}
\end{figure}


\subsection{Error observado}
Las pruebas anteriores eran realizadas con el uso normal que se le daba al variador de velocidad utilizando el panel digital frontal.

Luego, para comenzar con las pruebas de otros modos de funcionamiento del variador, se utilizó el potenciómetro frontal. Al utilizar este modo se originó error en la aceleración.   Al hacer las averiguaciones pertinentes, esto se debió a una configuración interna del variador: si se utiliza con el panel digital frontal, la curva de aceleración y desaceleración sigue una “s”, no realizando un cambio brusco en la velocidad del motor, en cambio, para el uso del potenciómetro u otro modo de funcionamiento la curva es lineal . Para revertir esto, se modificó el tiempo de aceleración y desaceleración a 20 segundos.

\begin{figure}[htbp]
	\centering
	\subfigure[Tipo s]{\includegraphics[width=60mm]{conS.png}}
	\subfigure[Tipo lineal]{\includegraphics[width=60mm]{sinS.png}}
	\caption{Curva aceleración y desaceleración} \label{fig:curva}
\end{figure}

\newpage