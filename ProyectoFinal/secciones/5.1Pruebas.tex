\subsection{Comparación de densidades} \label{cap:densidades}

Una vez que se tuvo seguridad con los sensores elegidos se procedió a realizar una placa con estos elementos para que no se desconecten y produzcan errores como solía suceder mientras estaban en la protoboard (Figura \ref{fig:sensoresa}). \\
\begin{figure}[H]
	\centering
	\includegraphics[scale=0.9]{placa_sensores.pdf}
	\captionof{figure}{Esquema de placa con sensores utilizados}
	\label{fig:sensoresa}
\end{figure}

A raíz de los resultados de varias mediciones, realizadas durante distintos días, y al realizar el contraste con el instrumento \textbf{TESTO 435} se decidió retirar de la caja a los sensores de temperatura y humedad para que luego el cálculo de la velocidad del aire no esté afectado por posibles cambios de estos, ya que esta caja utilizada generaba un ambiente distinto al real dentro del laboratorio. Al colocar el sensor del lado externo, se realizaron tomas de valores en distintos momentos. Con ambos conjuntos de datos se procedió a obtener los valores de la densidad calculados con la fórmula \ref{ec_den}. Estos datos están expresados en la tabla \ref{densicalc}. 


\begin{table}[b]
	\centering
	\begin{tabular}{ll|l|l|l|l|l|l|l|l|l|}
		\cline{3-11}
		\multicolumn{2}{c}{} & \multicolumn{2}{|c|}{\textbf{T [$^{\circ}$$C$]}} & \multicolumn{2}{c|}{\textbf{H [$\%$] }} & \multicolumn{2}{c|}{\textbf{P [$Pa$]}} &  \multicolumn{2}{c|}{\textbf{\begin{tabular}[c]{@{}c@{}}Dens. calculada\\ {[}$kg/m^3${]}\end{tabular}}} & \multicolumn{1}{c|}{\multirow{2}{*}{\textbf{$e_r$ [$\%$]}}} \\ \cline{1-10}
		\multicolumn{1}{|c|}{\textbf{Fecha}} & \multicolumn{1}{c|}{\textbf{Obs.}} & \multicolumn{1}{c|}{\textbf{I}} & \multicolumn{1}{c|}{\textbf{S}} & \multicolumn{1}{c|}{\textbf{I}} & \multicolumn{1}{c|}{\textbf{S}} & \multicolumn{1}{c|}{\textbf{I}} & \multicolumn{1}{c|}{\textbf{S}} & \multicolumn{1}{c|}{\textbf{I}} & \multicolumn{1}{c|}{\textbf{S}} & \multicolumn{1}{c|}{} \\ \hline
		\multicolumn{1}{|l|}{29-abr} & interior & 19,4 & 19,3 & 43,5 & 38,5 & 100400 & 100450 & 1,1916 & 1,1931 & 0,128 \\ \hline
		\multicolumn{1}{|l|}{14-may} & interior & 16,1 & 18,4 & 54,6 & 43,5 & 101590 & 101570 & 1,2195 & 1,2099 & 0,781 \\ \hline
		\multicolumn{1}{|l|}{14-may} & interior & 16,5 & 18,9 & 53,7 & 42,5 & 101559 & 101559 & 1,2173 & 1,2077 & 0,794 \\ \hline
		\multicolumn{1}{|l|}{17-jun} & exterior & 15,2 & 15,5 & 54,3 & 48,6 & 103100 & 103090 & 1,2418 & 1,2404 & 0,109 \\ \hline
		\multicolumn{1}{|l|}{17-jun} & exterior & 13,8 & 15,1 & 59,2 & 52,4 & 102970 & 102910 & 1,2463 & 1,2397 & 0,527 \\ \hline
		\multicolumn{1}{|l|}{07-jul} & interior & 17,6 & 18,8 & 43,3 & 34,8 & 100260 & 100261 & 1,1978 & 1,1933 & 0,375 \\ \hline
	\end{tabular}
	\caption{Comparación de densidades calculadas}
	\label{densicalc}
\end{table}




Referencias de la tabla:
\begin{itemize}
	\item \textbf{I}: Instrumentos- Datos obtenidos con los instrumentos del Laboratorio de fluidos.
	\item \textbf{S}: Sensores- Datos obtenidos a partir de la medición con los sensores utilizados en el proyecto.
	\item \textbf{interior}- Mientras se realizaron las mediciones el sensor SH21 se encontraba dentro de la caja dónde estaba el Arduino y la placa reguladora.
	\item \textbf{exterior}- Las mediciones se realizaron con el sensor SH21 en el lado exterior sin que fuera afectado por el calentamiento de la placa reguladora.
\end{itemize}


Al tomar como valor verdadero los valores de THP medidos con los instrumentos calibrados, se puede observar que el error relativo del cálculo de densidad es menor al 1\%.


%C:\Users\glori\Desktop\DANIELA\VISUAL_DANI\Proyecto_Final_Tunel\Pruebas22.03.21

\subsection{Estimación de la densidad ante cambios de Temperatura y Humedad}
Notamos conveniente realizar la comparación de las densidades calculadas ante las variaciones que se tenían de humedad y temperatura respecto a los datos tomados por los elementos calibrados. En la tabla \ref{densTH} se observa en las celdas internas los valores calculados de densidad para humedad de 38$\%$ y 40$\%$ y temperatura 19$^{\circ}$C y 21$^{\circ}$C, estos datos fueron elegidos al observar las máximas variaciones de los datos tomados con el sensor \textit{SH21} y el instrumento \textit{Testo 435}, al mantener la presión atmosférica constante.
(Los valores con fondo gris corresponden a la diferencia de densidades).

\begin{table}[H]
	\centering
	\begin{tabular}{ll|l|l|l|} 
		\cline{3-4}
		&                                    & \multicolumn{2}{c|}{\textbf{Temperatura}}                                                       & \multicolumn{1}{l}{}                            \\ 
		\cline{3-4}
		&                                    & \multicolumn{1}{c|}{\textbf{19°C}}             & \multicolumn{1}{c|}{\textbf{21°C}}             & \multicolumn{1}{c}{\textbf{}}                   \\ 
		\hline
		\multicolumn{1}{|c|}{\multirow{2}{*}{\textbf{Humedad}}} & \multicolumn{1}{r|}{\textbf{38\%}} & 1,1945758                                      & 1,1859383                                      & {\cellcolor[rgb]{0.816,0.816,0.816}}-0,0086376  \\ 
		\hhline{|~----|}
		\multicolumn{1}{|c|}{}                                  & \multicolumn{1}{r|}{\textbf{40\%}} & 1,1943783                                      & 1,1857162                                      & {\cellcolor[rgb]{0.816,0.816,0.816}}-0,0086621  \\ 
		\hline
		& \multicolumn{1}{c|}{\textbf{}}     & {\cellcolor[rgb]{0.816,0.816,0.816}}-0,0001976 & {\cellcolor[rgb]{0.816,0.816,0.816}}-0,0002221 & \multicolumn{1}{l}{}                            \\
		\hhline{~~--~}
	\end{tabular}
	\caption{Cálculo de la densidad ante cambios de temperatura y humedad}
	\label{densTH}
\end{table}

Si se calcula la velocidad del aire con la formula \ref{ec_aire} en conjunto con los datos de la tabla \ref{densTH} y se utiliza una diferencia de presión constante de 32$Pa$ se puede observar que la diferencias de velocidades son inferiores a 0,03m/s (tabla \ref{velTH}) ante un cambio de dos grados de temperatura por lo que no se ve necesario realizar una corrección de valores.


\begin{table}[H]
	\centering
	\begin{tabular}{ll|l|l|l|} 
		\cline{3-4}
		&                                    & \multicolumn{2}{c|}{\textbf{Temperatura}}                                                     & \multicolumn{1}{l}{}                           \\ 
		\cline{3-4}
		&                                    & \multicolumn{1}{c|}{\textbf{19°C}}            & \multicolumn{1}{c|}{\textbf{21°C}}            & \multicolumn{1}{c}{\textbf{}}                  \\ 
		\hline
		\multicolumn{1}{|c|}{\multirow{2}{*}{\textbf{Humedad}}} & \multicolumn{1}{r|}{\textbf{38\%}} & 7,3195288                                     & 7,3461357                                     & {\cellcolor[rgb]{0.816,0.816,0.816}}0,0266068  \\ 
		\hhline{|~----|}
		\multicolumn{1}{|c|}{}                                  & \multicolumn{1}{r|}{\textbf{40\%}} & 7,3201342                                     & 7,3468236                                     & {\cellcolor[rgb]{0.816,0.816,0.816}}0,0266894  \\ 
		\hline
		& \multicolumn{1}{c|}{\textbf{}}     & {\cellcolor[rgb]{0.816,0.816,0.816}}0,0006054 & {\cellcolor[rgb]{0.816,0.816,0.816}}0,0006879 & \multicolumn{1}{l}{}                           \\
		\hhline{~~--~}
	\end{tabular}
	\caption{Cálculo de velocidad del aire ante cambios de temperatura y humedad}
	\label{velTH}
\end{table}

\subsection{Contrastación de diferencia de presión}

El sensor \textit{MPX7002} tiene como salida una tensión proporcional a la diferencia de presión medida, por lo que fue necesario utilizar un \textit{ADS1115} para transformar estos valores, a través de una constante, en datos de presión digitalizados. (Sección \ref{sec:ads1115}).

Para realizar un contraste del valor de presión diferencial producido por la velocidad del aire dentro del túnel, en el tubo Pitot, se procedió a realizar mediciones con el instrumento calibrado \textit{AXD 650}. Al poseer fluctuaciones anteriormente mencionadas, debidas al flujo turbulento, y dado que el AXD650 no posee salida de datos (solo por display), se filmaron el instrumento y los datos obtenidos por nuestro sistema, para poder, con posterioridad, ver la correlación entre ambos. Los dos videos se unieron y se realizaron diversas pausas para tomar al mismo tiempo ambos valores leídos. Como resultado se obtuvo la tabla \ref{difpres}, luego se generó un gráfico de puntos con las muestras tomadas y se observó que el error es mayor para diferencias de presión mayores a 100 Pa (Figura \ref{fig:condifpres}).

\begin{figure}[H]
	\centering
	\includegraphics[scale=0.5]{condifpres.jpg}
	\captionof{figure}{Contraste de valores de presión diferencial del MPX7002}
	\label{fig:condifpres}
\end{figure}

\begin{table}[H]%diferencia de presion
		\centering
		\begin{tabular}{|r|r|r}
			\cline{1-2}
			\multicolumn{2}{|c|}{Diferencia de presión {[}Pa{]}} & \multicolumn{1}{c}{} \\ \hline
			\multicolumn{1}{|c|}{\textbf{AXD650}} & \multicolumn{1}{c|}{\textbf{MPX7002}} & \multicolumn{1}{c|}{\textbf{E$_r$}} \\ \hline
			14 & 13,51 & \multicolumn{1}{r|}{3,50\%} \\ \hline
			14 & 13,76 & \multicolumn{1}{r|}{1,71\%} \\ \hline
			14 & 13,76 & \multicolumn{1}{r|}{1,71\%} \\ \hline
			21,3 & 20,88 & \multicolumn{1}{r|}{1,97\%} \\ \hline
			21,5 & 20,88 & \multicolumn{1}{r|}{2,88\%} \\ \hline
			25,3 & 24,8 & \multicolumn{1}{r|}{1,98\%} \\ \hline
			25,4 & 24,8 & \multicolumn{1}{r|}{2,36\%} \\ \hline
			28,9 & 28,51 & \multicolumn{1}{r|}{1,35\%} \\ \hline
			29,3 & 28,8 & \multicolumn{1}{r|}{1,71\%} \\ \hline
			33,4 & 32,5 & \multicolumn{1}{r|}{2,69\%} \\ \hline
			34,1 & 32,88 & \multicolumn{1}{r|}{3,58\%} \\ \hline
			35,9 & 34,8 & \multicolumn{1}{r|}{3,06\%} \\ \hline
			38,1 & 37,38 & \multicolumn{1}{r|}{1,89\%} \\ \hline
			44,2 & 43,88 & \multicolumn{1}{r|}{0,72\%} \\ \hline
			45,5 & 44,5 & \multicolumn{1}{r|}{2,20\%} \\ \hline
			45,9 & 44,88 & \multicolumn{1}{r|}{2,22\%} \\ \hline
			49,4 & 48,51 & \multicolumn{1}{r|}{1,80\%} \\ \hline
			49,9 & 48,51 & \multicolumn{1}{r|}{2,79\%} \\ \hline
			54 & 53,26 & \multicolumn{1}{r|}{1,37\%} \\ \hline
			57,7 & 56,2 & \multicolumn{1}{r|}{2,60\%} \\ \hline
			62,1 & 58,13 & \multicolumn{1}{r|}{6,39\%} \\ \hline
			68 & 67,13 & \multicolumn{1}{r|}{1,28\%} \\ \hline
			73,4 & 76,6 & \multicolumn{1}{r|}{-4,36\%} \\ \hline
			74,3 & 75,26 & \multicolumn{1}{r|}{-1,29\%} \\ \hline
			81,1 & 79,01 & \multicolumn{1}{r|}{2,58\%} \\ \hline
			85,5 & 82,76 & \multicolumn{1}{r|}{3,20\%} \\ \hline
			86,4 & 86,1 & \multicolumn{1}{r|}{0,35\%} \\ \hline
			86,8 & 82,63 & \multicolumn{1}{r|}{4,80\%} \\ \hline
			104 & 99,26 & \multicolumn{1}{r|}{4,56\%} \\ \hline
			106,5 & 104,13 & \multicolumn{1}{r|}{2,23\%} \\ \hline
			106,8 & 104,5 & \multicolumn{1}{r|}{2,15\%} \\ \hline
			107,8 & 104,63 & \multicolumn{1}{r|}{2,94\%} \\ \hline
			118,8 & 120,13 & \multicolumn{1}{r|}{-1,12\%} \\ \hline
			120,4 & 120,6 & \multicolumn{1}{r|}{-0,17\%} \\ \hline
			120,5 & 117,38 & \multicolumn{1}{r|}{2,59\%} \\ \hline
			124,4 & 119 & \multicolumn{1}{r|}{4,34\%} \\ \hline
			140,1 & 136,6 & \multicolumn{1}{r|}{2,50\%} \\ \hline
			147,2 & 143,8 & \multicolumn{1}{r|}{2,31\%} \\ \hline
			152,7 & 149,38 & \multicolumn{1}{r|}{2,17\%} \\ \hline
		\end{tabular}
	\caption{Contraste de los valores de presión diferencial.}
	\label{difpres}

\end{table}

\subsection{Contrastación de velocidades}
Como prueba final se decidió realizar la contrastación de la velocidad estimada contra la velocidad de otro dispositivo. Para esto, se procedió a colocar en el interior del túnel un anemómetro digital \textbf{Avm-01 -\textit{ Prova}} perteneciente al laboratorio. A medida que se realizaban pruebas, se realizó tanto la filmación del anemómetro como la de la pantalla para luego unir ambos videos y poder tomar valores al mismo tiempo (Figura \ref{fig:capt}).

Estos valores fueron volcados a la tabla \ref{difvel} y luego graficados para observar posibles errores(Figura \ref{fig:vel_muestras}).

\begin{table}[H]
	\centering
	\begin{tabular}{|r|r|r}
		\cline{1-2}
		\multicolumn{2}{|c|}{velocidad {[}m/s{]}} & \multicolumn{1}{l}{} \\ \hline
		\multicolumn{1}{|c|}{\textbf{estimada}} & \multicolumn{1}{c|}{\textbf{anemómetro}} & \multicolumn{1}{c|}{\textbf{E$_r$}} \\ \hline
		4,6 & 4,7 & \multicolumn{1}{r|}{2,13\%} \\ \hline
		4,7 & 4,7 & \multicolumn{1}{r|}{0,00\%} \\ \hline
		4,8 & 4,7 & \multicolumn{1}{r|}{-2,13\%} \\ \hline
		5 & 5 & \multicolumn{1}{r|}{0,00\%} \\ \hline
		5,9 & 6 & \multicolumn{1}{r|}{1,67\%} \\ \hline
		6 & 6,1 & \multicolumn{1}{r|}{1,64\%} \\ \hline
		6,1 & 6,2 & \multicolumn{1}{r|}{1,61\%} \\ \hline
		7,8 & 7,8 & \multicolumn{1}{r|}{0,00\%} \\ \hline
		7,9 & 8 & \multicolumn{1}{r|}{1,25\%} \\ \hline
		8,9 & 9,2 & \multicolumn{1}{r|}{3,26\%} \\ \hline
		9,1 & 9,3 & \multicolumn{1}{r|}{2,15\%} \\ \hline
		11,9 & 12,4 & \multicolumn{1}{r|}{4,03\%} \\ \hline
		12 & 12,3 & \multicolumn{1}{r|}{2,44\%} \\ \hline
		12 & 12,4 & \multicolumn{1}{r|}{3,23\%} \\ \hline
		12,2 & 12,4 & \multicolumn{1}{r|}{1,61\%} \\ \hline
		12,3 & 12,5 & \multicolumn{1}{r|}{1,60\%} \\ \hline
		13,7 & 14,5 & \multicolumn{1}{r|}{5,52\%} \\ \hline
		13,8 & 14,4 & \multicolumn{1}{r|}{4,17\%} \\ \hline
		13,8 & 14,4 & \multicolumn{1}{r|}{4,17\%} \\ \hline
		13,9 & 14,3 & \multicolumn{1}{r|}{2,80\%} \\ \hline
		14 & 14,4 & \multicolumn{1}{r|}{2,78\%} \\ \hline
		14,2 & 14,5 & \multicolumn{1}{r|}{2,07\%} \\ \hline
		14,3 & 14,9 & \multicolumn{1}{r|}{4,03\%} \\ \hline
		14,7 & 15,6 & \multicolumn{1}{r|}{5,77\%} \\ \hline
		14,8 & 15,7 & \multicolumn{1}{r|}{5,73\%} \\ \hline
		14,9 & 15,5 & \multicolumn{1}{r|}{3,87\%} \\ \hline
		14,9 & 15,8 & \multicolumn{1}{r|}{5,70\%} \\ \hline
		15 & 15,7 & \multicolumn{1}{r|}{4,46\%} \\ \hline
		15,1 & 16,2 & \multicolumn{1}{r|}{6,79\%} \\ \hline
		15,3 & 16,3 & \multicolumn{1}{r|}{6,13\%} \\ \hline
	\end{tabular}
	\caption{Contraste de los valores de velocidad.}
\label{difvel}
\end{table}


\begin{figure}[H]
	\centering
	\includegraphics[scale=0.5]{vel_muestras.png}
	\captionof{figure}{Contraste de valores de velocidad estimada}
	\label{fig:vel_muestras}
\end{figure}


Al examinar la tabla y el gráfico se observa errores mayores a medida que las velocidades se incrementan.

\begin{figure}[H]
	\centering
	\includegraphics[scale=0.5]{captura.jpg}
	\captionof{figure}{Captura de pantalla del video utilizado para la realización del contraste de velocidad}
	\label{fig:capt}
\end{figure}


