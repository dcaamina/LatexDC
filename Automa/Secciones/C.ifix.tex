Para realizar la pantalla de supervisión, control y adquisición de datos operador (SCADA) se utilizó el software iFix perteneciente al grupo \textbf{General Electric}.\\
El sistema HMI creado (Figura \ref{fig:scada1}) se dividió en las siguientes secciones:
\begin{itemize}
	\item Esquemático del circuito hidráulico físico con las variables de presión y caudal en tiempo real.
	\item Valores de funcionamiento del motor obtenidos por el variador de velocidad.
	\item Alarmero, dónde se observa de forma visual valores críticos alcanzados en el sistema.
	\item Indicador de modo de funcionamiento físico o remoto.
	\item Modo de control a lazo abierto o lazo cerrado.
	\subitem Para el modo de lazo cerrado se creó una ventana individual para cada sistema de presión y caudal.
	\item Botón para gráficos en tiempo real dónde se divide según la variable a observar.
	\item Botón para ingresar a la pantalla donde se observa datos históricos y se puede generar un archivo \textit{.csv} con la información de la variable elegida en un determinado período de tiempo.
\end{itemize} 

\begin{figure}[htb]
	\centering
	\includegraphics[scale=0.5]{scada2.png}
	\captionof{figure}{Pantalla SCADA}
	\label{fig:scada1}
\end{figure}


\subsubsection{Configuración iFix o Configuracion MBE?}
Para realizar la configuración de cada ícono con su respectiva variable se debió crear un MBE dónde se estipula la dirección y los mapas de memorias que luego serán utilizadas por el DataBase. 


\paragraph{MBE}
\subsubsection{DataBase}

\paragraph{Lista de direcciones utilizadas}
Poner lista que esta en drive
\paragraph{Pruebas mediante ModSim}
Para realizar pruebas previas a la implementación final se utilizó el programa ModSim, donde se generó los distintos mapas de memoria utilizados y allí se podían modificar variables para observarlas en SCADA.

\begin{figure}[htb]
	\centering
	\includegraphics[scale=0.5]{modsim1.png}
	\captionof{figure}{ModSim}
	\label{fig:modsim1}
\end{figure}

\subsubsection{Distintas pantallas}


\subsection{Alarmas}
Dentro de la pantalla principal es posible observar el alarmero. Estas alarmas están compuestas por las variables de la siguiente \fcolorbox{red}{yellow}{tabla} con sus respectivos valores y prioridades.\\
\fcolorbox{red}{yellow}{Poner lista que esta en drive}
\subsection{iHistorian}
\begin{figure}[htb]
	\centering
	\includegraphics[scale=0.5]{scada3.png}
	\captionof{figure}{Pantalla SCADA}
	\label{fig:scada3}
\end{figure}

\subsection{Direcciones}
Poner acá las de ifix y unity? como la utilizada en drive?