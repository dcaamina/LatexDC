Como se nombró en el objetivo, se busca realizar el control de caudal o presión de un sistema hidráulico. Para esto fue necesario realizar la implementación de un banco de pruebas que cuente de tres partes (Figura \ref{fig:bancofull}).
\begin{itemize}
	\item Soporte para el motor y variador de velocidad,  diseñado y construido por el profesor Gerardo Arthz. A estos elementos se realizó las correspondientes conexiones, y se agregó elementos adicionales: 3 señales luminosas, llave selectora de dos puntos para seleccionar el modo de comunicación, llave selectora de tres puntos (encendido y sentido del motor) y un pulsador de parada de emergencia (Figura \ref{fig:banco}(a)).
	Tanto el motor y los elementos adicionales fueron cableados (Figura \ref{fig:banco}(b)) hacia las borneras del variador de velocidad y se tuvo en cuenta para esto las características y funciones del bornero de control proporcionado por el manual del variador de velocidad\cite{InstaManual}. 
	
	\item Soporte para una bomba en desuso, de características no conocidas con su bobinado quemado.
	\item Circuito hidráulico, que incluye un tanque, válvulas y sensores de caudal y presión.
\end{itemize}




\begin{figure}[htb]
	\centering
	\includegraphics[scale=0.2]{bancofull.png}
	\captionof{figure}{Banco de pruebas completo}
	\label{fig:bancofull}
\end{figure}


\begin{figure}[H]
	\centering
	\subfigure[]{\includegraphics[angle=-90,width=60mm]{banc1 (1)}}
	\subfigure[]{\includegraphics[angle=-90,width=60mm]{images/banc1 (2)}}
	\caption{Banco de Pruebas} \label{fig:banco}
\end{figure}



\subsubsection{Presupuesto}
\fcolorbox{red}{yellow}{falta lo de la bomba}
\url{https://docs.google.com/spreadsheets/d/1mFoNvgJXUdL2bNnspaBJ_wS5fFA1y1c8fTO9Rfod7H0/edit#gid=0}

\newpage