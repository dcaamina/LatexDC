
\clearpage
\newpage

\section{Mejoras futuras}
En un futuro se espera que alumnos de la carrera realicen mejoras en el banco de pruebas, por ejemplo:
\begin{itemize}
	\item Mejorar la distancia entre los sensores y los accesorios del sistemas, como válvulas o codos para que el fluido no se torne turbulento.
	\item Implementar sistemas sonoros o visuales de las alarmas.
	\item Generar una página web para observar los datos en tiempo real y/o manejar de forma remota.
	\item Realizar otro anexo no hidráulico para colocar al motor- variador y generar un nuevo banco de pruebas.
	\item Generar nuevas formas de perturbación a los sistemas.
	\item Implementar un sistema para controlar presión o caudal por medio de válvulas proporcionales.
	\item Realizar perturbaciones controladas y repetibles con válvulas proporcionales.
	\item Reemplazar bomba por una en mejor estado.
	\item Realizar pruebas de caudal y presión a mayor frecuencia.
\end{itemize}

\newpage
\section{Conclusiones}
Se concluye que el banco de pruebas construido es una herramienta útil para alumnos de las carreras de ingeniería que sigan ramas orientadas al control automatizado, ya que se tiene la posibilidad de generar perturbaciones en el sistema y observar distintas respuestas.

Se generó un sistema SCADA dónde se puede observar diversas variables en tiempo real y realizar estudios de ellas a través de datos históricos. Además, en la pantalla, se puede observar distintos tipos de anomalías causadas por el variador y los instrumentos utilizados en el proyecto, para facilitar la detección de errores y poder solucionarlos adecuadamente.

Se logró tener control eficaz sobre tres variables distintas mediante la variación de la frecuencia del motor, dónde esta es la única acción de control en el banco de pruebas.

Finalmente, la realización del proyecto tuvo un gran aporte para consolidar los conocimientos obtenidos durante la cursada de la materia \textit{Automatización Industrial} y también para el crecimiento personal y profesional.



\newpage