Un banco de pruebas de un motor cuenta con un punto de apoyo donde se conecta el motor y sus componentes mecánicos, ademas dentro de esta plataforma existe un sistema de medición que posee sensores, variador y PLC para los procedimientos de prueba.
Un banco de pruebas puede ser un prototipo de un gran desarrollo industrial o simplemente un banco formado para realizar pruebas educativas. \\

El objetivo de este trabajo final para la cátedra de Automatización Industrial es construir un banco de pruebas para ser utilizado por cualquier persona dentro el laboratorio de Automatización y Control. Se espera generar un banco de pruebas que cuente con:
\begin{itemize}
    \item Motor trifásico 1,5kW (Altium)\textit{-Proporcionado por la cátedra-}
    \item PLC (Schneider - M340) \textit{-Proporcionado por la cátedra-}
    \item Variador de velocidad (Schneider - ATV312 ) \textit{-Proporcionado por la cátedra-}
    \item Panel de control
        \subitem Botón de emergencia
        \subitem Encendido/ apagado
        \subitem Potenciómetro para variar velocidad
        \subitem Display para observar velocidad
        \subitem Alarmas visuales
    \item HMI
        \subitem Alarmas
        \subitem Información en tiempo real
        \subitem Histórico de datos
        \subitem Control general del banco
\end{itemize}

\newpage