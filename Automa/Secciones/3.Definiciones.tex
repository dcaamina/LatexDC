\subsection{¿Qué es un banco de pruebas?}

Un banco de pruebas de un motor cuenta con un punto de apoyo donde se conecta el motor y sus componentes mecánicos, ademas dentro de esta plataforma existe un sistema de medición que posee sensores y variador y PLC para los procedimientos de prueba.
Un banco de pruebas puede ser un prototipo de un gran desarrollo industrial o simplemente un banco formado para realizar pruebas educativas. 


\subsection{¿Qué es un variador?}
Un variador de velocidad (VSD, por sus siglas en inglés Variable Speed Drive) es utilizado para controlar la velocidad de giro de un motor. \\
Para regular las revoluciones, se debe tener en cuenta las características del motor, ya que este tiene una curva propia de funcionamiento. Para seguir esta curva se emplea un variador pudiendo este ser utilizado junto con ventiladores, bombas, elevadores, portones, etc generando en estos elementos control de aceleración, frenado, seguridad, control del torque y operaciones que mejoran la eficiencia energética.

\subsection{¿Qué es un Motor asincrónico trifásico?}
Los motores eléctricos son máquinas que transforman la energía eléctrica en
movimiento (energía cinética). Estos aparatos se componen, básicamente, del rotor y de un estator donde tiene bobinas inductoras desfasadas entre sí 120°.

\subsection{¿Qué es ModBus?}
Modbus es un protocolo de comunicaciones situado en el nivel 7 del Modelo OSI, basado en la arquitectura maestro/esclavo o cliente/servidor, diseñado en 1979 por Modicon para su gama de controladores lógicos programables (PLCs). Convertido en un protocolo de comunicaciones estándar en la industria, es el que goza de mayor disponibilidad para la conexión de dispositivos electrónicos industriales.
%http://microelecblog.blogspot.com/2013/12/configuracion-atv312-para-red-modbus.html

\subsection{¿Qué es CanOpen?}